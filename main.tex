\documentclass{article}
\usepackage{graphicx}
\usepackage{booktabs}
\usepackage{parskip}
\usepackage{tcolorbox}
\usepackage[fleqn]{amsmath}
\usepackage{multicol}
\usepackage{amsmath}
\usepackage{multicol}
\usepackage{float}
\usepackage{amssymb}
\usepackage{gensymb}
\usepackage[inline]{enumitem}
\usepackage{titling}
\usepackage[margin=0.5in]{geometry}
\setlength{\droptitle}{-1in}
\usepackage{array}
\usepackage{amsmath}
\usepackage{natbib}
\usepackage{hyperref}
\newcommand{\commandnote}[1]{
    \begin{tcolorbox}[
        standard jigsaw,
        title=Note,
    ]
        #1
    \end{tcolorbox}
}
\newcommand{\minititle}[1]{
\subsubsection*{#1}
}

\title{\vspace{1ex} Data Eng. Notes \vspace{-1ex}}
\author{ibrahim.nasser@fau.de}
\date{}
\begin{document}
\maketitle
\section*{Introduction and Basic Data Types}
We call the data \textbf{Tabular} when there are no modelled dependencies between attributes, for example, demographic attributes such as age, gender, ZIP code, etc. (also called \textit{Nondependency-Oriented Data}). Otherwise it is \textbf{Non-Tabular}, e.g. social networks, time series, etc.

\textbf{Matrix Representation of Data}

A set  $X = \{ X_i \mid  i \in \{1 \dots n \}\}, $ with $n$ records (samples) is a $d$-dimensional dataset iff each sample $X_i$ is a set of $\{ x_j \mid j \in \{ 1 \dots d\} \}$ attributes (features). $X$ is tabular if it is invariant w.r.t shuffling of samples and features. Each feature $x_j$ has its own domain $\mathcal{D}_j$

\textbf{Quantitative vs. Categorical}
A variable $x$ is quantitative (numeric) if its domain $\mathcal{D}_x$ is numeric. Otherwise, Categorical.
\textit{Examples (Q): }age, weight, height, BMI, Date of Birth.
\textit{Examples (C): }name, gender, country, ZIP Code, weather, ID, day.

\textbf{Nominal vs. Ordinal}
A categorical variable $x$ is ordinal if its domain $\mathcal{D}_x$ has a natural ordering. Otherwise, Nominal.
\textit{Examples (N): }weather, name, gender, country, ZIP, ID, day
\textit{Examples (O): }heat level, textual gpa.

\textbf{Finite vs. Infinite}
A variable $x$ has a finite domain iff $|\mathcal{D}_x| = N , N \in \mathbb{N}$. Otherwise, Infinite.
\textit{Examples (F): }age (years), country, ZIP, ID, gender, day.
\textit{Examples (I): }BMI, height, Date of Birth.
\commandnote{All categorical variables have finite domains, not the other way around.}

\textbf{Discrete vs. Cont.}
A Quantitative variable $x$ is continuous iff $\forall z,y \in \mathcal{D}_x \exists w \in \mathcal{D}_x, z<w<y$. Otherwise, Discrete.
\textit{Examples (D): }age (years, months, days, hours, etc).
\textit{Examples (C): }age (unitless, number), Date of Birth (point in cont. time), BMI.
\commandnote{By \textbf{rounding} quantitative data, we can transform cont. domains into discrete ones.}
\commandnote{Age is quantitative finite discrete if it is computed as whole years, months, days, hours. However, it is quantitative infinite continuous it is computed as precise value including fractions}
\commandnote{Date of Birth is quantitative infinite continuous since it is a point in a continuous endless time}
\textbf{Binary} We call a variable $x$ binary iff $|\mathcal{D}_x|=2$

\textbf{Temporal} We call a variable $x$ temporal iff $\mathcal{D}_x$ represents time points or intervals. \textit{Examples:} day, month, Date of Birth


\minititle{Encoding}
Data Encoding refers to the technique of converting data into a form that allows it to be properly used by different systems.

\minititle{Binning}
Binning is an encoding technique that is a function $f : \mathcal{D} \to \{1 \dots K\}$

\textbf{Example: Equal-Width Binning:} Size (width) of each bin is calculated as $W = \frac{\text{Max}(x)-\text{Min}(x)}{K}$ where $K$ is the number of bins.

\minititle{One-Hot Encoding}
To mitigate the problem of label encoding for nominal variables.\\
\textbf{How?} Create a fixed-size vector with size = $|\text{unique}(x)|$, where each position corresponds to a unique category value. Assign a \texttt{1} to the position representing the category and \texttt{0}s elsewhere.

\textbf{Example:}
Suppose $\text{unique}(x) = \{\text{Red}, \text{Green}, \text{Blue}\}$

\begin{itemize}
    \item Red $\rightarrow$ [1, 0, 0]
    \item Green $\rightarrow$ [0, 1, 0]
    \item Blue $\rightarrow$ [0, 0, 1]
\end{itemize}

\commandnote{
    One-hot encoding avoids the problem of implying ordinal relationships.
    However, it increases dimensionality significantly, especially when the number of categories is large (curse of dimensionality).
}

\minititle{Cyclic Encoding}
Some categorical variables are \textit{ordinal} and have a natural \textit{cyclic} structure. A classic example is the months of the year:
\[
\mathcal{D}_x = \{\text{Jan}, \text{Feb}, \dots, \text{Dec}\}
\]

This variable has both an order (Jan $<$ Feb $<$ ... $<$ Dec) and a cyclic relationship (Dec is followed by Jan).

To encode this properly, we use the index \( i \) of each category in the ordered list, where \( i = 1, 2, \dots, k \), and \( k \) is the total number of categories.

\textbf{Encoding Function:}
\[
\text{enc}(c_i) = (x_i, y_i)
\]
\[
x_i = \cos\left(\frac{2\pi(i-1)}{k}\right),\quad y_i = \sin\left(\frac{2\pi(i-1)}{k}\right)
\]

This maps each category to a unique point on the unit circle, preserving both order and cyclicity.

\commandnote{
    Cyclic encoding is useful when the first and last categories are conceptually adjacent (e.g., December and January). This is not possible with standard label or one-hot encoding.
}
\textbf{Optional: Normalize to Unit Square}
\[
\text{enc}(c_i) = \left(\frac{x_i + 1}{2},\ \frac{y_i + 1}{2}\right)
\]
This scaled version maps points to the square $[0, 1] \times [0, 1]$, which can be useful when input normalization is required for machine learning models. Note that this transformation alters the original unit circle geometry.

\commandnote{
    Use raw unit circle encoding when preserving angular distance is important. Use the normalized version when the model expects features in the range $[0, 1]$.
}   

\minititle{Non-Tabular Data}
Such as Spatial data, images, time series, string, graphs.

A set $X = \{x_i \mid i \in \{1 \dots n\}\}$ is a $d$-dimensional \textbf{spatial} dataset with $n$ samples if each sample $x_i$ contains a set of $\{ x_j \mid j \in \{ 1 \dots d\} \}$  features AND each data point $x_{ij}$ is associated with a specific spatial location $l$.

A spatial location $l$ can be a point $(l_x, l_y) \in \mathbb{R}^2$ (2D spatial data) or $(l_x, l_y, l_z) \in \mathbb{R}^3$ (3D spatial data), etc.

\minititle{Tokenization (Character-Level)}

Tokenization is the process of converting raw text into smaller units called tokens. In character-level tokenization, each unique character from the corpus is treated as a token.

\textbf{Example:} Consider the corpus consisting of a single sentence:  
\texttt{"hi ai"}

\begin{itemize}
    \item Unique characters: \texttt{\{h, i, \space, a\}}  
    \item Assign token IDs: \texttt{h:0,\ i:1,\space:2,\ a:3}
    \item Tokenized sentence: \texttt{"hi ai"} $\rightarrow$ \texttt{[0, 1, 2, 3, 1]}
\end{itemize}

Each character in the sentence is replaced by its corresponding token ID.

\minititle{Graphs}

A graph is a mathematical structure used to model pairwise relations between objects.

\begin{itemize}
    \item A graph \( G \) is defined as \( G = (V, E) \), where:
    \begin{itemize}
        \item \( V \) is a set of \textit{vertices} (or \textit{nodes}).
        \item \( E \subseteq V \times V \) is a set of \textit{edges}.
    \end{itemize}
\end{itemize}

\textbf{Types of Graphs:}
\begin{itemize}
    \item \textbf{Undirected Graph:}  
    An edge \( (u, v) \in E \) implies a bidirectional connection:  
    \[
    (u, v) \in E \Rightarrow (v, u) \in E
    \]

    \item \textbf{Directed Graph (Digraph):}  
    Edges have direction:  
    \[
    (u, v) \in E \not\Rightarrow (v, u) \in E
    \]
\end{itemize}

\minititle{Graph Representations}

\textbf{Adjacency Matrix:}

A \( |V| \times |V| \) matrix \( A \), where:
\[
A[u][v] = 
\begin{cases}
1 & \text{if } (u,v) \in E \\
0 & \text{otherwise}
\end{cases}
\]

\begin{itemize}
    \item \textbf{Space consumption:} \( \mathcal{O}(|V|^2) \)
    \item \textbf{Edge access:} \( \mathcal{O}(1) \)
    \item \textbf{Neighbor iteration:} \( \mathcal{O}(|V|) \)
\end{itemize}

\textbf{Adjacency List:}

Each vertex \( u \in V \) maintains a list of its neighbors.

\begin{itemize}
    \item \textbf{Space consumption:} \( \mathcal{O}(|V| + |E|) \)
    \item \textbf{Edge access:} \( \mathcal{O}(|V|) \) (worst-case search)
    \item \textbf{Neighbor iteration:} \( \mathcal{O}(\deg(u)) \), where \( \deg(u) \) is the degree of vertex \( u \)
\end{itemize}

    \textbf{Weighted Graphs:}  

    In some graphs, each edge \( (u, v) \in E \) is associated with a numerical value called a \textit{weight}, often representing cost, distance, capacity, etc.
    
    \begin{itemize}
        \item For weighted graphs, the edge set becomes:  
        \[
        E \subseteq V \times V \times \mathbb{R}
        \]
        or we define a weight function:  
        \[
        w : E \rightarrow \mathbb{R}
        \]
        \item In the adjacency matrix, \( A[u][v] \) stores the weight instead of a binary 0 or 1.
        \item In the adjacency list, each neighbor can be stored along with its edge weight as a tuple: \( (v, w(u, v)) \).
    \end{itemize}    
\end{document}